% arara: xelatex: { synctex: yes, shell: yes }
\documentclass{../ist-thesis}

\begin{document}

\begin{center}
	{\Huge\bfseries A classe \texttt{ist-thesis}}
	\par\bigskip
	{\Large Classe LaTeX de tese não oficial do Instituto Superior Técnico}
	\par\bigskip\smallskip
	Daniel Lopes de Schiffart \\
	\url{https://github.com/ekspek/ist-thesis}
\end{center}

\bigskip\bigskip

\begin{center}
	\section*{Introdução}
\end{center}
Esta classe tem como objetivo preparar e formatar qualquer documento \LaTeX{} numa tese de mestrado de acordo com os regulamentos do Instituto Superior Técnico.

\tableofcontents

\clearpage

\chapter{\textit{Quick Start} -- Os Básicos}

Este capítulo está dedicado a cobrir os básicos para colocar a classe a funcionar em qualquer instalação de \LaTeX{}.

\chapter{Comandos e Ambientes}

Este capítulo contém a documentação oficial dos comandos e ambientes implementados na classe utilizáveis pelo utilizador e as suas funções e dependências dentro do documento que as utiliza. Também incluídas estão algumas modificações feitas a documentos base de \LaTeX{} para funcionarem corretamente com as funções implementadas.

Estes comandos podem ser separados em duas categorias, relevantes para a capa e informação do documento, na secção \ref{sec:metadata}, e para a estrutura e secções especiais, na secção \ref{sec:spec}.

\section{Capa e Dados de Tese}\label{sec:metadata}

Os comandos nesta secção são pertinentes aos dados do contexto da tese, como título, autor, e orientadores, por exemplo, e à aplicação destes dados na capa da tese.

\section{Secções Especiais}\label{sec:spec}

A descrição de comandos pertinentes à introdução de secções especiais na tese encontram-se nesta secção.

\chapter{Implementação da Classe}

Este capítulo é dedicado à descrição de como a classe e todas as suas funções foram implementadas dentro do contexto de \TeX{} e \LaTeX{}, com especial atenção às \textit{packages} utilizadas. Tem por isso um cariz mais técnico e não é necessário a sua leitura para qualquer utilização da classe.

\section{Tipo de Letra}

Em contraste com o tipo de letra de raíz de qualquer documento \LaTeX{}\footnotemark{}, \textit{Computer Modern}, o tipo de letra especificado pelo guia de preparação da dissertação, \textit{Arial} é muito distinto, mas igualmente (ou mais reconhecível). No entanto, um dos problemas é que este tipo de letra é muito mais omnipresente em sistemas com \textit{Windows} do que no restantes, como também na internet. É, por isso, um tipo de letra que não encontrou implementação nativa em \LaTeX{}.
\footnotetext{Neste caso referimo-nos a documentos utilizando as classes base de \LaTeX{}, como \texttt{article}, \texttt{report}, ou \texttt{book}, entre outras.}

No entanto, \textit{Arial} é baseado no tipo de letra \textit{Helvetica}, que é facilmente disponível online em vários formatos e codificações. Os dois tipos de letra são praticamente indistinguíveis, sendo o design de \textit{Arial} baseado fortemente em \textit{Helvetica} na altura da sua criação. Esta segunda opção, pela sua facilidade de obtenção, já encontrou várias implementações em \LaTeX{}.

Para o tipo de letra base deste projeto escolhemos o tipo de letra \href{https://ctan.org/pkg/tex-gyre-heros}{\TeX{} Gyre Heros}, que está incluído em qualquer distribuição de \LaTeX{}. A razão desta escolha foi a sua facilidade de implementação e relativa solidez em várias misturas de propriedades, desde \textbf{negrito} a \textit{itálico} à {\bfseries\itshape mistura dos dois}, como também a caracteres estrangeiros, nomeadamente caracteres utilizados na língua portuguesa. A grande lacuna deste tipo de letra é a falta de tipografia matemática, uma lacuna que será discutida na secção \ref{sec:mathfont}.

\subsection{\textit{Monoespaço}}

\end{document}
