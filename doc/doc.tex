% arara: xelatex: { synctex: yes, shell: yes }
\documentclass{../ist-thesis}

\begin{document}

\begin{center}
	{\Huge\bfseries A classe \texttt{ist-thesis}}
	\par\bigskip
	{\Large Classe LaTeX de tese não oficial do Instituto Superior Técnico}
	\par\bigskip\smallskip
	Daniel Lopes de Schiffart \\
	\url{https://github.com/ekspek/ist-thesis}
\end{center}

\bigskip\bigskip

\begin{center}
	\section*{Introdução}
\end{center}
Esta classe tem como objetivo preparar e formatar qualquer documento \LaTeX{} numa tese de mestrado de acordo com os regulamentos do Instituto Superior Técnico.

\tableofcontents

\clearpage

\chapter{\textit{Quick Start} -- Os Básicos}

Este capítulo está dedicado a cobrir os básicos para colocar a classe a funcionar em qualquer instalação de \LaTeX{}.

\chapter{Comandos e Ambientes}

Este capítulo contém a documentação oficial dos comandos e ambientes implementados na classe utilizáveis pelo utilizador e as suas funções e dependências dentro do documento que as utiliza. Também incluídas estão algumas modificações feitas a documentos base de \LaTeX{} para funcionarem corretamente com as funções implementadas.

Estes comandos podem ser separados em duas categorias, relevantes para a capa e informação do documento, na secção \ref{sec:metadata}, e para a estrutura e secções especiais, na secção \ref{sec:spec}.

\section{Capa e Dados de Tese}\label{sec:metadata}

Os comandos nesta secção são pertinentes aos dados do contexto da tese, como título, autor, e orientadores, por exemplo, e à aplicação destes dados na capa da tese.

\section{Secções Especiais}\label{sec:spec}

A descrição de comandos pertinentes à introdução de secções especiais na tese encontram-se nesta secção.

\end{document}
